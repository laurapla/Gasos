%(text; description of the need that is being covered / advantages or disadvantages of your approach / usefulness of the project / critical elements of the designing process / previous experiences a.s.o.)

\section{Justification}
\subsection{Need that is being covered}
Conservation equations of mass, momentum and energy appear in a variety of cases. Most thermal and engineering problems require to solve these equations to achieve the desired result. However, they usually do not have an analytical solution, so a computational approach is often necessary. A huge amount of cases have been solved in the recent years, but there are still other problems that need to be studied and developed.
\newline
A better understanding on the computational resolution of the conservation equations can lead to better results in the numerical simulations. As a consequence, they could improve the actual knowledge in a variety of subjects, such as the temperature variation inside an engine or the way the air moves inside the lunges. Furthermore, they could also be used in the optimization of different engineering systems; for example, more efficient wings for future airplanes.

\subsection{Advantatges and drawbacks}
The main advantage of the approach explained in the scope is that the study of the computational resolution is started from basic cases and its difficulty is upgraded with every case of fluid dynamics that is proposed. That way, the comprehension on the developed simulations is higher, which makes the codes more reliable. As previously mentioned, the simulation codes are being developed from zero. This is an advantage because no previous errors are going to be introduced on the program, but it is also a drawback because its development could take some time.
\newline
\newline
Anyhow, this project can be useful in the study of new engineering and thermal problems that need to be solved using the conservation equations of mass, momentum and energy; and can lead to other new studies of computational resolution of these equations.
