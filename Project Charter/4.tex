%(text; description of the need that is being covered / advantages or disadvantages of your approach / usefulness of the project / critical elements of the designing process / previous experiences a.s.o.)

\section{Justification}
Conservation equations of mass, momentum and energy are necessary to solve thermal and engineering problems. However, they usually do not have an analytical solution, so a computational approach is often necessary. A huge amount of cases have been solved in the recent years, but there are still other problems that need to be studied and developed, in particular some practical applications.

The main advantage of the approach explained in the scope is that the study of the computational resolution is started from basic cases and its difficulty is upgraded with every case of fluid dynamics that is proposed. That way, the comprehension on the developed simulations is higher and the codes are more reliable. As previously mentioned, the simulation codes are being developed from zero. This is an advantage because no previous errors are going to be introduced on the program, but it is also a disadvantage because its development could take some time.

Anyhow, this project can be useful in the study of new engineering and thermal problems that need to be solved using the conservation equations of mass, momentum and energy; and can lead to other new studies of computational resolution of these equations.

