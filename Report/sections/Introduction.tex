\chapter{Introduction}
\section{Aim}
The main objective of this paper is to provide knowledge in the computational resolution of the fundamental equations of fluid dynamics and mass and heat transfer by developing simulation codes. A second objective would be to apply the developed and verified codes to a specific case.

\section{Scope}
First, some basic cases concerning the equations of mass, momentum and energy are going to be solved in order to learn the fundamentals of the mathematical formulation and the computational and programming techniques that are going to be needed to develop the whole study. With the help of these cases, some simulation codes are going to be developed.

A second part of this paper is going to be the application of the knowledge acquired to a practical case.

In order to accomplish the objectives mentioned above, these are the following tasks to be developed:
\begin{itemize}
	\item Previous research of the state of the art.
	\item Theoretical approach of the fluid dynamics behind all the cases and study of the mathematical formulation that should be applied.
	\item Development of the necessary numerical simulation tools. All the codes will need to be validated to ensure they are correct.
	\item Application of the acquired knowledge in simulation codes to an specific system.
	\item Analysis of the results.
	\item Conclusions.
\end{itemize}

\section{Requirements}
\begin{itemize}
	\item Codes must be developed in C or C++.
	\item No external libraries or solvers can be used.
	\item Codes must be in a single file and compile with no errors.
	\item Codes must run without any input.
	\item Codes should be able to be executed in a normal computer.
	\item Simulations should provide realistic results.
\end{itemize}

\section{Justification}
\subsection{Identification of the need}
Conservation equations of mass, momentum and energy define the motion of fluids. Most thermal and engineering problems require to solve these equations to achieve the desired result. However, they are coupled differential equations, which means they are difficult to solve. Except for a few simplified cases, they usually do not have an analytical solution, so a numerical approach is often necessary. A huge amount of cases have been solved in the recent years, but there are still other problems that need to be studied and developed.

Since these equations need a numerical resolution in the majority of cases, the knowledge of computational techniques is essential to improve the simulations in accuracy and efficiency. A better understanding on the computational resolution of the conservation equations can lead to better results in the numerical simulations and with less computational cost. As a consequence, the actual knowledge in a variety of subjects could be improved, such as the temperature variation inside an engine or the way the air moves in the respiratory system. Furthermore, it could also lead to an optimization of different engineering systems; for example, more efficient wings for future airplanes.

\subsection{Advantatges and drawbacks}
The main advantage of the approach explained in the scope is that the study of computational resolution starts from basic cases and its difficulty is upgraded with every case of fluid dynamics that is proposed. That way, the comprehension on the developed simulations is higher, which makes the codes more reliable. However, the simulation codes are being developed from zero. This is an advantage because no previous errors are going to be introduced on the program, but it is also a drawback because its development could take some time.

Anyhow, this project can be useful in the study of new engineering and thermal problems that need to be solved using the conservation equations of mass, momentum and energy; and can lead to other new studies of computational resolution of these equations.