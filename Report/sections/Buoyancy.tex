\chapter{Differentially heated cavity}

The differentially heated cavity problem or buoyancy-driven cavity problem is a two dimensional problem based on free convection. It comprises a cavity with the vertical walls at different temperatures and adiabatic horizontal walls \cite{DeVahlDavis1983a, DeVahlDavis1983}. The difference of temperatures between the two vertical walls causes convection.

\begin{figure}[h]
	\centering
	\includegraphics[scale=0.5]{Buoyancy/BuoyancyDriven}
	\caption[General scheme of the differentially heated problem]{General scheme of the differentially heated problem. Extracted from \cite{Ferziger2002}}
\end{figure}

\section{Natural convection}
The equations that are going to be discretized are the two-dimensional equations of mass, momentum and energy for a Newtonian fluid with constant properties. However, some approximations have to be done for the momentum equation:
\begin{equation}
\rho_{0}\frac{\partial\vec{v}}{\partial t}+\rho_{0}\left(\vec{v}\cdot\nabla\right)\vec{v}=-\nabla p+\mu\nabla^{2}\vec{v}+\rho\vec{g}
\end{equation}
As it can be seen, mass forces have been added to the momentum equation, because in natural convection gravity is not negligible. Another important remark is that the fluid is considered incompressible, but this approximation is not applied in the gravitational term, because it is this term that causes the flow. However, as a simplification, the Boussinesq approximation is applied \cite{Bergman2011,Ghiaasiaan2011}:
\begin{equation}
\rho=\rho_{0}\left(1-\beta\left(T-T_{0}\right)\right)
\end{equation}
where $\beta$ is the volumetric thermal expansion coefficient, a thermodynamic property that measures the variation of the density as a function of the temperature at a constant pressure.
Introducing the Boussinesq approximation in the momentum equation, the final equations for natural convection are:
\begin{equation}
\nabla\cdot\vec{v}=0
\label{NatConvContinuity}
\end{equation}
\begin{equation}
\rho_{0}\frac{\partial\vec{v}}{\partial t}+\rho_{0}\left(\vec{v}\cdot\nabla\right)\vec{v}=-\nabla p_{d}+\mu\nabla^{2}\vec{v}-\rho_{0}\beta\left(T-T_{0}\right)\vec{g}
\label{NatConvMomentum}
\end{equation}
\begin{equation}
\rho c_{p}\left(\frac{\partial T}{\partial t}+\vec{v}\cdot\nabla T\right)=\lambda\nabla^{2}T
\label{NatConvEnergy}
\end{equation}

\subsection{Non-dimensional equations}
In order to have a simpler analysis, it is convenient to use non-dimensional variables. Depending on the problem, they can be defined by several different ways. In this section, taking into account the most important variables of a free convection problem, the dimensionless variables are defined as it is expressed below:
\begin{equation}
\vec{x}^{*}=\frac{\vec{x}}{L}
\end{equation}
\begin{equation}
\vec{v}^{*}=\frac{\vec{v}}{\frac{\lambda}{\rho Lc_{P}}}
\end{equation}
\begin{equation}
t^{*}=\frac{t}{\frac{\rho L^{2}c_{P}}{\lambda}}
\end{equation}
\begin{equation}
p^{*}=\frac{p}{\frac{1}{\rho}\left(\frac{\lambda}{c_{P}L}\right)^2}
\end{equation}
\begin{equation}
T^{*}=\frac{T-T_{cold}}{T_{hot}-T_{cold}}
\end{equation}
Inserting these expressions in the equations \ref{NatConvContinuity}, \ref{NatConvMomentum} and \ref{NatConvEnergy}, the non-dimensional equations for natural convection are obtained. In order to simplify the notation, the indexes $^{*}$ have been removed, but the variables that are represented in the following section are the dimensionless ones.
\begin{equation}
\nabla\cdot\vec{v}=0
\end{equation}
\begin{equation}
\frac{\partial\vec{v}}{\partial t}+\left(\vec{v}\cdot\nabla\right)\vec{v}=-\nabla p+Pr\nabla^{2}\vec{v}-PrRaT\vec{u}_{g}
\end{equation}
\begin{equation}
\frac{\partial T}{\partial t}+\vec{v}\cdot\nabla T=\nabla^{2}T
\end{equation}

The dimensionless numbers that appear are the Prandtl and Rayleigh numbers, defined as:
\begin{equation}
Pr\equiv\frac{c_{P}\mu}{\lambda}
\end{equation}
\begin{equation}
Ra\equiv\frac{\rho^{2} g\beta\Delta TL^{3}c_{P}}{\mu\lambda}
\end{equation}
where $\Delta T=T_{hot}-T_{cold}$.

\section{Application of the fractional step method}
The resolution of this problem can be done using the fractional step method described in chapter \ref{FractionalStepM}. In the case of the momentum equation its resolution is done as it is explained in the mentioned section. But the expression of the equation is slightly different. Rewritting the momentum equation as it was done:
\begin{equation}
\frac{\partial\vec{v}}{\partial t}=R_{v}\left(\vec{v}\right)-\nabla p
\end{equation}
where
\begin{equation}
R_{v}\left(\vec{v}\right)=-\left(\vec{v}\cdot\nabla\right)\vec{v}+Pr\nabla^{2}\vec{v}-PrRaT\vec{u}_{g}
\end{equation}

The other important issue is the introduction of the energy equation. It has to be discretized according to this method.
\begin{equation}
\frac{\partial T}{\partial t}=R_{T}\left(T\right)
\end{equation}
where
\begin{equation}
R_{T}\left(T\right)=-\vec{v}\cdot\nabla T+\nabla^{2}T
\end{equation}
As done with the momentum equation, the Adams-Bashforth scheme is applied for the temporal discretization:
\begin{equation}
\frac{T^{n+1}-T^{n}}{\Delta t}=\frac{3}{2}R_{T}\left(T^{n}\right)-\frac{1}{2}R_{T}\left(T^{n-1}\right)
\end{equation}

\section{Discretization}
The spatial discretization of the domain is the one described by the figure \ref{staggered}: a Cartesian grid with a staggered mesh.

The discretization of the momentum equation is very similar to that described in the fractional step method chapter but adding the gravity term in the case of the vertical equation:
\begin{multline}
	R_{v}\left(u\right)V_{P}=\left[Pr_{e}\frac{u_{E}-u_{P}}{d_{EP}}A_{e}+Pr_{n}\frac{u_{N}-u_{P}}{d_{NP}}A_{n}-Pr_{w}\frac{u_{P}-u_{W}}{d_{WP}}A_{w}-Pr_{s}\frac{u_{P}-u_{S}}{d_{SP}}A_{s}\right] \\
	-\left[\left(u\right)_{e}u_{e}A_{e}+\left(v\right)_{n}u_{n}A_{n}-\left(u\right)_{w}u_{w}A_{w}-\left(v\right)_{s}u_{s}A_{s}\right]
\end{multline}
\begin{multline}
	R_{v}\left(v\right)V_{P}=\left[Pr_{e}\frac{v_{E}-v_{P}}{d_{EP}}A_{e}+Pr_{n}\frac{v_{N}-v_{P}}{d_{NP}}A_{n}-Pr_{w}\frac{v_{P}-v_{W}}{d_{WP}}A_{w}-Pr_{s}\frac{v_{P}-v_{S}}{d_{SP}}A_{s}\right] \\
	-\left[\left(u\right)_{e}v_{e}A_{e}+\left(v\right)_{n}v_{n}A_{n}-\left(u\right)_{w}v_{w}A_{w}-\left(v\right)_{s}v_{s}A_{s}\right]-PrRaTV_{P}
\end{multline}

In the case of the energy equation a similar approach is used. Temperature, like pressure, is evaluated in the nodes, not in the faces like the velocities.
\begin{equation}
\int_{\Omega}R_{T}d\Omega=-\int_{\Omega}\vec{v}\nabla Td\Omega+\int_{\Omega}\nabla^{2}Td\Omega
\end{equation}
Applying the Gauss Theorem:
\begin{equation}
R_{T}V_{P}=-\int_{\partial\Omega}\vec{v}\cdot\vec{n} TdS+\int_{\partial\Omega}\nabla T\vec{n}dS
\end{equation}
\begin{multline}
	R_{T}V_{P}=\left[\frac{T_{E}-T_{P}}{d_{PE}}A_{e}+\frac{T_{N}-T_{P}}{d_{PN}}A_{n}-\frac{T_{P}-T_{W}}{d_{PW}}A_{w}-\frac{T_{P}-T_{S}}{d_{PS}}A_{s}\right] \\
	-\left[\left(u\right)_{e}T_{e}A_{e}+\left(v\right)_{n}T_{n}A_{n}-\left(u\right)_{w}T_{w}A_{w}-\left(v\right)_{s}T_{s}A_{s}\right]
\end{multline}
The temperature in the faces is evaluated using the central differencing scheme, as it was done for the velocities.