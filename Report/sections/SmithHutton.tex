\chapter{Smith-Hutton problem}
The Smith-Hutton problem is a two-dimensional steady convection-diffusion problem, represented in figure \ref{SHscheme}. In the problem, the density is constant, and the velocity field is known.
\begin{figure}[h!]
	\centering
	\includegraphics[scale=0.65]{SmithHutton/SmithHutton}
	\caption{General scheme of the Smith-Hutton problem}
	\label{SHscheme}
\end{figure}

The velocity is given by equations \ref{SHuvel} and \ref{SHvvel}:
\begin{equation}
u\left(x,y\right)=2y\left(1-x^{2}\right)
\label{SHuvel}
\end{equation}
\begin{equation}
v\left(x,y\right)=-2x\left(1-y^{2}\right)
\label{SHvvel}
\end{equation}
And the prescribed boundary conditions for the variable $\phi$ are described in equation \ref{BoundCond}:
\begin{equation}
\begin{aligned}
\phi &=1+\tanh\left(\alpha\left(2x+1\right)\right),&&y=0; x\in\left(-1,0\right) \left(inlet\right) \\
\frac{\partial\phi}{\partial y} &=0,&&x=0; y\in\left(-1,0\right) \left(outlet\right) \\
\phi &=1-\tanh\left(\alpha\right),&&\left(elsewhere\right)
\end{aligned}
\label{BoundCond}
\end{equation}
where $\alpha=10$.

\section{Discretization}
The spatial discretization is the one explained in section \ref{DiscretizationSH}, and the temporal scheme used is the implicit one, as explained in the same section.

\begin{table}[h]
	\centering
	\begin{tabular}{ |c|c|c|c| }
		\hline
		$N_{x}$ & $N_{y}$ & $\delta$ & Source term \\ \hline
		$200$ & $100$ & $10^{-9}$ & $0$ \\ \hline
	\end{tabular}
\caption{Numerical parameters of the Smith-Hutton problem}
\end{table}

\section{Boundary conditions}
To apply the boundary conditions to the problem, some modifications on the coefficients defined by equation \ref{DiscretizedEquGCDE} have to be done. The easiest one is for the points in the left, top and right sides of the domain, in which the value of $\phi$ is defined.

In the bottom it is necessary to distinguish between the inlet and the outlet. A similar approach to that of the left, top and right side is used to determine the coefficients of the points in the inlet. However, in the outlet the only condition is that the derivative of $\phi$ in the vertical direction is equal to zero. The following reasoning leads to:
\begin{equation}
\frac{\partial\phi}{\partial y}\approx\frac{\phi_{N}-\phi_{P}}{\Delta y}=0
\end{equation}
\begin{equation}
\phi_{P}=\phi_{N}
\end{equation}
The implementation of the boundary conditions in the problem is done with the discretization coefficients listed in table \ref{BoundaryCondSH}.
\begin{table}[h]
	\centering
	\begin{tabular}{ |c|c|c|c| }
		\hline
		Coefficients & Left, top and right & Inlet (bottom) & Outlet (bottom) \\ \hline
		$a_{E}$ & $0$ & $0$ & $0$ \\ \hline
		$a_{W}$ & $0$ & $0$ & $0$ \\ \hline
		$a_{N}$ & $0$ & $0$ & $1$ \\ \hline
		$a_{S}$ & $0$ & $0$ & $0$ \\ \hline
		$a_{P}$ & $1$ & $1$ & $1$ \\ \hline
		$b_{P}$ & $1-\tanh\left(\alpha\right)$ & $1+\tanh\left(\alpha\left(2x+1\right)\right)$ & $0$ \\ \hline
	\end{tabular}
\caption{Discretization coefficients of the boundary points}
\label{BoundaryCondSH}
\end{table}


\section{Algorithm}
The algorithm of resolution is very similar to that used in the four materials problem. The main difference is that in the Smith-Hutton problem the resolution ends when the variable $\phi$ reaches a steady state. The schematic resolution is displayed below:
\begin{figure}[h!]
	\includegraphics[scale=0.17]{SmithHutton/algorithm}
\end{figure}

\section{Results}
Some of the results are plotted in the following section, but due to the amount of information, not all of them are in the report. To see more results refer to Attachment A.

Since the velocity and the dimensions are constant, the parameter $\rho/\Gamma$ is somehow equivalent to the Péclet number. In figure \ref{ResSH10}, when the Péclet number is low, all the methods have similar results, and there is almost no error compared to the reference values. However, as the Péclet number increases, the error increases, as it can be seen in figure \ref{ResSH1000} and figure \ref{ResSH1000000}.

\begin{figure}[h]
	\centering
	\input{SmithHutton/10}
	\caption{Comparative of the different methods $\rho/\Gamma=10$}
	\label{ResSH10}
\end{figure}
\begin{figure}[h]
	\centering
	\input{SmithHutton/1000}
	\caption{Comparative of the different methods $\rho/\Gamma=10^{3}$}
	\label{ResSH1000}
\end{figure}
\begin{figure}[h]
	\centering
	\input{SmithHutton/1000000}
	\caption{Comparative of the different methods $\rho/\Gamma=10^{6}$}
	\label{ResSH1000000}
\end{figure}
It is important to notice that in the cases in which $\rho/\Gamma=10^{3}$ and $\rho/\Gamma=10^{6}$ CDS diverges, and no results can be obtained with this method. It is not a surprise, because the central differencing scheme tends to diverge in some cases.

It is also important to understand the variation of the variable $\phi$ in the whole domain. For a low Péclet number (figure \ref{ResuSH10}), the variation is very gradual. But as it increases (figures \ref{ResuSH1000} and \ref{ResuSH1000000}), the change is more abrupt, and there is almost no variation except for the centre bottom zone. The shape of the variation becomes more symmetrical as the Péclet number increases.
\begin{figure}[h]
	\centering
	\input{SmithHutton/R10}
	\caption{Representation of the whole domain for $\rho/\Gamma=10$ (UDS)}
	\label{ResuSH10}
\end{figure}
\begin{figure}[h]
	\centering
	\input{SmithHutton/R1000}
	\caption{Representation of the whole domain for $\rho/\Gamma=10^{3}$ (UDS)}
	\label{ResuSH1000}
\end{figure}
\begin{figure}[h]
	\centering
	\input{SmithHutton/R1000000}
	\caption{Representation of the whole domain for $\rho/\Gamma=10^{6}$ (UDS)}
	\label{ResuSH1000000}
\end{figure}

As a conclusion, it can be stated that when convergence dominates over diffusion (low Péclet numbers), velocity distributes the properties of the inlet to a great part of the domain. However, as diffusion becomes more important (the Péclet number increases), the influence of the inlet becomes less important.