\chapter{Economical and environmental considerations}

\section{Budget}
The economical cost of this study is divided in four sections:
\begin{itemize}
	\item Human resources: The study was developed by an aerospace engineering student during 300 hours.
	\item Software:It refers to the programs used to develop the study: \textit{TeXstudio 2.12.4}, \textit{Dev-C++ 5.11}, \textit{Matlab}, \textit{Microsoft Excel 2010} and \textit{Mendeley Desktop 1.17.9}.
	\item Hardware: One of the requirements of the study was that the developed codes had to be able to be executed in a normal computer. The hardware refers to the laptop used in the development of this study.
	\item Power consumption: It refers to the electricity used by the computer in the development of the paper.
\end{itemize}
To see more details refer to the Budget document.


\section{Environmental impact}
In the making of this paper different cases were studied: conduction through a wall, the movement of a fluid inside a cavity, natural convection... With the use of computational methods these studies only required a computer to be done. As a consequence, it avoided an experimental release of the problems and the use of an appropriate installation to perform them. This aspect reduces considerably the environmental impact that the study could have had.

As for the resources used, the main tool was a computer and the electricity needed for its operation. However, these resources in the making of this paper do not generate an additional environmental impact, because they are commonly used in a student's everyday life.

In short, the environmental impact of this study was minimum.

\section{Planning}
To see the planning refer to Annex C.